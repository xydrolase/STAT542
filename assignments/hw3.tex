\documentclass[letter]{article}
\usepackage[margin=1in]{geometry}
\usepackage{amsmath, amssymb}
\usepackage{enumerate}
\usepackage{fancyhdr}

\newcommand{\cX}{\cal{X}}
\newcommand{\cY}{\cal{Y}}

\pagestyle{fancy}
\fancyhead[L]{STAT 542 HW3}
\fancyhead[R]{Xin Yin}

\begin{document}
    % Casella & Berger 1.49, 1.52, 1.53, 2.2, 2.6, 2.8, 2.9   
    \section*{1.49}
    Given that cdf $F_X$ is stochastically larger than $F_Y$, \emph{i.e.}  
    $F_X(t) \leq F_Y(t)$ for all $t$ and $F_X(T) < F_Y(T)$ for some $t$,
    we have,
    \begin{eqnarray*}
    F_X(t) = P(X \leq t) \leq F_Y(t) = P(Y \leq t), \quad \text{for all}~t\\
    F_X(t) = P(X \leq t) < F_Y(t) = P(Y \leq t), \quad \text{for some}~t
    \end{eqnarray*}
    Because cdf's are nondecreasing functions, we have,
    \begin{eqnarray*}
    1-P(X > t) \leq 1-P(Y > t), \quad \text{for all}~t\\
    P(X > t) \geq P(Y > t), \quad \text{for all}~t.
    \end{eqnarray*}
    Similarly, we can have,
    \[
    P(X>t) > P(Y>t) \quad \text{for some}~t.
    \]

    \section*{1.52}
    If $g(x)$ is a pdf if and only if,
    \begin{enumerate}[(i)]
        \item $g(x) \geq 0$ for all $x$,
        \item $\int_{-\infty}^{\infty} g(x) dx = 1$
    \end{enumerate}
    Since $g(x) = f(x)/[1-F(x_0)] I(x \geq x_0)$, and $f(x)$ is a pdf, 
    $F(x_0) < 1$, we know that, $f(x)/[1-F(x_0)] \geq 0$.
    Also,
    \[
    \int_{x_0}^\infty g(x)dx = \frac{1}{1-F(x_0)} \int_{x_0}^\infty f(x)dx
    = \frac{P(x_0 \leq x \leq \infty)}{1-F(x_0)} = \frac{1-F(x_0)}{1-F(x_0)} =
    1
    \]

    Therefore, $g(x)$ is a pdf.

    \section*{1.53}
    For $F_Y(y) = P(Y \leq y) = 1-\frac{1}{y^2}, 1 \leq y < \infty$,
    \begin{enumerate}[(a)]
    \item
    \begin{eqnarray*}
    \lim_{y \to 1} = 1 - \frac{1}{1} = 0 \\
    \lim_{y \to \infty} = 1 - \frac{1}{\infty} = 1 \\
    \frac{d}{dy} F_Y(y) = \frac{2}{y^3} > 0, \text{for}~1 \leq y < \infty
    \end{eqnarray*}
    Plus, $F_Y(y)$ is continuous on $\cY$, $F_Y(y)$ is then a cdf.
    \item 
    \[
    f_Y(y) = \frac{d}{dy} F_Y(y) = \frac{2}{y^3}
    \]
    \item For r.v. $Z$,
    \begin{align*}
    F_Z(z) & = P_Z(Z \leq z) = P_Y\left(10(Y-1) \leq z\right) \\
    & = P_Y \left(Y \leq \frac{z + 10}{10}\right) = F_Y(\frac{z+10}{10}) \\
    & = 1 - \frac{100}{(z+10)^2}, \quad 0 \leq Z < \infty
    \end{align*}
    \end{enumerate}

    \section*{2.2}
    \begin{enumerate}[(a)]
    \item $Y = X^2, 0 < x < 1$ implies 
    $y=g(x) = x^2, g^{-1}(y) = \sqrt{y}, 0 < y < 1$. $g$ is a monotone function
    on $\cX$. Applying theorem (2.1.5), we have,
    \[
    f_Y(y) = f_X(g^{-1}(y)) \left| \frac{d g^{-1}(y)}{dy} \right| 
    = \frac{1}{2\sqrt{y}}, \quad 0 < y < 1
    \]
    \item Given $Y = -\log X, 0 < x < 1$, we have,
    \[ g^{-1}(y) = e^{-y}, \quad 0 < y < \infty \]
    And $g$ is monotone function on $\cX$.
    Therefore, 
    \[
    f_Y(y) = f_X(e^{-y}) \left| \frac{d e^{-y}}{dy} \right |
    = \frac{(n+m+1)!}{n!m!} e^{-(n+1)y} (1-e^{-y})^m, \quad 0 < y < \infty
    \]
    
    \item Given $Y = e^X, 0 < x < \infty$, we have,
    \[ g^{-1}(y) = \log y, \quad 1 < y < \infty \]
    $g$ is again monotone on $\cX$, then we can have,
    \begin{align*}
    f_Y(y) & = f_X(\log y) \left| \frac{d \log y}{dy} \right |
    = \frac{1}{\sigma^2} \log y e^{-\log (2y)/(2\sigma^2)} \frac{1}{y} \\
    & = \frac{1}{\sigma^2} y^{-(1+\frac{1}{2 \sigma^2})} \log y, \quad 1 < y <
    \infty.
    \end{align*}
    \end{enumerate}

    \section*{2.6}
    \begin{enumerate}[(a)]
    \item Using Theorem (2.1.8), we can partition $\cX$, which is $\mathbf{R}$
    into $A_0 = \{0\}, A_1 = (-\infty, 0), A_2 = (0, \infty)$. $y=g(x) = |x|^3$
    is monotone on $A_1$ and $A_2$. In addition,
    \[
    g_1^{-1}(y) = -y^{1/3}, g_2^{-1}(y) = y^{1/3},
    \] 
    are both continuous and derivative on $\cY$, which is $(0, \infty)$.

    Therefore, applying Theorem (2.1.8), we can have,
    \begin{align*}
    f_Y(y) & = f_X\left(g_1^{-1}(y)\right)\left|\frac{d}{dy} g_1^{-1}(y)\right| +
    f_X\left(g_2^{-1}(y)\right)\left|\frac{d}{dy} g_2^{-1}(y)\right| \\
    & = \frac{e^{-y/3}}{3y^{2/3}}, \quad y \in (0, \infty)
    \end{align*}

    We can show that,
    \begin{align*}
    F_Y(\infty) & = \int_0^\infty \frac{e^{-y/3}}{3 y^{2/3}} =
    - \int_0^\infty e^{-y^{1/3}} d(-y^{1/3}) \\
    & = -e^{y/3}\rvert^\infty_0 = 1
    \end{align*}

    \item Again, we partition $\cX$ into 
    $A_0 = \{0\}, A_1 = (-1, 0), A_2 = (0,1)$, $g(x)$ is monotone on 
    $A_i, i = 1,2$. 
    We have,
    \[
    g_1^{-1}(y) = -\sqrt{1-y}, g_2^{-1}(y) = \sqrt{1-y}, \quad y \in (0,1).
    \]
    $g_i^{-1}(y),\quad i=1,2$ is continuous derivative on $A_1, A_2$. 
    Hence, 
    \begin{align*}
    f_Y(y) & = f_X(-\sqrt{1-y})\left| \frac{d}{dy} -\sqrt{1-y} \right|
    + f_X(\sqrt{1-y})\left| \frac{d}{dy} \sqrt{1-y} \right| \\
    & = \frac{3}{16\sqrt{1-y}}\left[(1-\sqrt{1-y})^2 + (1+\sqrt{1-y})^2\right]
    \end{align*}

    To show that pdf $f_Y(y)$ is a legitimate one, we can integrate the pdf , by letting $t = \sqrt{1-y}, \quad t \in (0, 1)$, as
    \begin{align*}
    \int_0^1 \frac{3}{16\sqrt{1-y}}\left[(1-\sqrt{1-y})^2 + (1+\sqrt{1-y})^2\right] & = -\frac{3}{8} \int_1^0 [(1-t)^2+(1+t)^2] dt \\
    & = \frac{3}{4} \int_0^1 1+ t^2 dt = 3/4 + 3/4*\frac{t^3}{3}|^1_0 = 1
    \end{align*}

    \item We partition $\cX = (-1, 1)$ into 
    $A_0 = \{0\}, A_1 = (-1, 0), A_2 = (0,1)$. $g_1(x) = 1-x^2$ and 
    $g_2(x) = 1-x$ are respectively monotone on $A_1, A_2$. 
    Also,
    \[
    g^{-1}_1(y) = -\sqrt{1-y}, g^{-1}_2(y) = 1-y, \quad y \in (0,1)
    \]
    are both continuous derivative on $\cY$. 
    
    So, we can write pdf of r.v. $Y$ as,
    \[
    f_Y(y) = f_X(-\sqrt{1-y})\left|\frac{d}{dy} -\sqrt{1-y}\right|
    + f_X(1-y)\left|\frac{d(1-y)}{dy}\right| =
    \frac{3}{16\sqrt{1-y}}(1-\sqrt{1-y})^2 + \frac{3}{8}(2-y)^2, \quad y \in (0,1)
    \]
    To integrate $f_Y(y)$ over $\cY$,
    \begin{align*}
    F_Y(1) & = \frac{3}{16}\int_0^1 (1-\sqrt{1-y})^2\frac{1}{\sqrt{1-y}} dy 
    + \frac{3}{8} (2-y)^2 dy \\
    & = -\frac{3}{8} \left[\int_0^1 (1-t)^2 d(1-t) + \int_0^1 (2-y)^2
    d(2-y)\right] \\
    & = -\frac{1}{8} \left[ (1-t)^3 \rvert^1_0 + (2-y)^3\rvert^1_0\right]
    = 1
    \end{align*}
    \end{enumerate}

    \section*{2.8}
    \begin{enumerate}[(a)]
    \item $F_X(x)$ does not have derivative at $x=0$. Using the definition, 
    \[
    F^{-1}_X(y) = \inf\{x: F(x) \geq y\} = \begin{cases}
    -\infty & y=0\\
    -\log(1-y) & 0 < y < 1\\
    \infty & y = 1
    \end{cases}
    \]
    \item $\lim{x \uparrow 0} e^x/2 = \frac{1}{2}$.
    And $F^{-1}_X(\frac{1}{2}) = \inf\{x: F(x) \geq \frac{1}{2}\} = 0$.

    \[
    F^{-1}_X(y) = \begin{cases}
    -\infty & y = 0 \\
    \log 2y & 0 < y \leq \frac{1}{2} \\
    1-\log(2-2y) &  \frac{1}{2} < y < 1 \\
    \infty & y = 1
    \end{cases}
    \]

    \item $\lim_{x \uparrow 0} e^x/4 = 1/4$, 
    $\lim_{x \downarrow 0} 1-(e^{-x}/4) = 3/4$. Thus, $F_X(x)$ is discontinous
    at $x=0$.
    
    Therefore, for all $y \in [\frac{1}{4}, \frac{3}{4})$, 
    $F^{-1}_X(1/4 \leq y < 3/4) = 0$.

    \[
    F^{-1}_X(y) = \begin{cases}
    -\infty & y = 0 \\
    \log 4y & 0 < y < \frac{1}{4} \\
    0 & \frac{1}{4} \leq y < \frac{3}{4} \\
    -\log (4-4y) & \frac{3}{4} < y < 1 \\
    \infty & y = 1
    \end{cases}
    \]
    \end{enumerate}
    
    \section*{2.9}
    Using Theorem (2.1.10), we know that, if a monotone function exists for $X$
    such that $Y=u(X)$ is a uniform distribution on $(0,1)$, the function
    $u(X)$ should be the cdf of $X$.
    For r.v. X, it has pdf,
    \[
    f_X(x) = \frac{x-1}{2} I(1 < x < 3)
    \]
    We can then have it's cdf as,
    \[
    F_X(x) = \int_x \frac{x-1}{2} dx = \frac{1}{2} \int_x x-1 d(x-1) = \frac{1}{4}
    (x-1)^2 + C
    \]
    Because $\lim_{x \to 3} F_X(x) = 1$, we know that $C=0$ and 
    that $F_X(x) = \frac{(x-1)^2}{4}$. And $F_X(x)$ is monotone on $\cX$

    Therefore, $u(X) = \frac{(x-1)^2}{4}$ is the montone function such that
    $Y=u(X)$ has a uniform distribution.
\end{document}
