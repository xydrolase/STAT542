\documentclass[letterpaper]{article}
\usepackage[margin=1in]{geometry}
\usepackage{amsmath, amssymb}
\usepackage{enumerate}
\usepackage{fancyhdr}

\pagestyle{fancy}
\fancyhead[L]{STAT 542 HW9}
\fancyhead[R]{Xin Yin}

\newcommand{\intii}{\int_{-\infty}^\infty}
\newcommand{\intzi}{\int_0^\infty}
\renewcommand{\arraystretch}{1.5}

\begin{document}
    \section*{4.16}
    \begin{enumerate}
    \item[(c)]
    Let $Z=X+Y, \quad z = 2, 3, \dots$, so $Y=Z-X$.
    \[
    P(X=x, X+Y=z) = P(X=x, Y=z-x) = p(1-p)^{x-1} p(1-p)^{z-x-1} = p^2(1-p)^{z-2}, \quad z = 2, 3, \dots
    \]
    \end{enumerate}
    \section*{4.23}
    \begin{enumerate}[(a)]
    \item
    For the joint pdf of $(X, Y)$,
    \begin{align*}
    f_{X,Y}(x,y) & = \frac{\Gamma(\alpha+\beta)}{\Gamma(\alpha)\Gamma(\beta)} x^{\alpha-1} (1-x)^{\beta-1} \frac{\Gamma(\alpha+\beta+\gamma)}{\Gamma(\alpha+\beta)\Gamma(\gamma)} y^{\alpha+\beta-1} (1-y)^{\gamma-1}
    & 0 < x < 1, \quad 0 < y < 1.
    \end{align*}
    If $U=XY, 0 < u < 1, \quad V=Y, 0 < v < 1$, we have, 
    \[
    X=U/V, \quad Y=V.
    \]
    So the Jacobian is,
    \[
    J = \begin{vmatrix}
    \frac{\partial u/v}{\partial u} & \frac{\partial u/v}{\partial v} \\
    \frac{\partial v}{\partial u} & \frac{\partial v}{\partial v} \\
    \end{vmatrix} = 1/v
    \]

    Then, the joint density for $(U, V)$ is,
    \begin{align*}
    f_{U,V}(u, v) & = f_{X,Y}(uv, v)|J| \\
    & = \frac{\Gamma(\alpha+\beta)}{\Gamma(\alpha)\Gamma(\beta)} \left(\frac{u}{v}\right)^{\alpha-1} (1-\frac{u}{v})^{\beta-1} \frac{\Gamma(\alpha+\beta+\gamma)}{\Gamma(\alpha+\beta)\Gamma(\gamma)} v^{\alpha+\beta-1} (1-v)^{\gamma-1} \frac{1}{v} \\
    \\
    f_U(u) & = \int_0^1 f(u,v)dv \\
    & = \frac{\Gamma(\alpha+\beta+\gamma)}{\Gamma(\alpha)\Gamma(\beta)\Gamma(\gamma)} u^{\alpha-1} \int_0^1 (v-u)^{\beta-1}(1-v)^{\gamma-1} dv \\
    & \text{Let $y = \frac{v-u}{1-u}, \quad dv = d\big((1-u)y+u\big)$},\\
    & = \frac{\Gamma(\alpha+\beta+\gamma)}{\Gamma(\alpha)\Gamma(\beta+\gamma)} u^{\alpha-1} (1-u)^{\beta+\gamma-1} \int_0^1 \frac{\Gamma(\beta+\gamma)}{\Gamma(\beta)\Gamma(\beta)} y^{\beta-1} (1-y)^{\gamma-1} dy \\
    & = \frac{\Gamma(\alpha+\beta+\gamma)}{\Gamma(\alpha)\Gamma(\beta+\gamma)} u^{\alpha-1} (1-u)^{\beta+\gamma-1} 
    \end{align*}
    So, $U \sim \text{Gamma}(\alpha, \beta+\gamma)$.
    \item Similarly, given 
    \[
    \begin{cases}
    U = XY,  \quad 0 < u < 1, \\
    V = X/Y, \quad u < v < \frac{1}{u} < \infty
    \end{cases}
    \]
    , we have, 
    \[
    X = \sqrt{uv}, \quad Y = \sqrt{\frac{u}{v}}.
    \]
    The Jacobian is,
    \[
    J = \begin{vmatrix}
    \frac{\partial \sqrt{uv}}{\partial u} & \frac{\partial \sqrt{uv}}{\partial v} \\
    \frac{\partial \sqrt{u/v}}{\partial \sqrt{u/v}} & \frac{\partial v}{\partial v} \\
    \end{vmatrix} = -\frac{1}{2v}
    \]

    \begin{align*}
    f_{U,V}(u,v) & = f_{X,Y}(\sqrt{uv}, \sqrt{u/v})|J| \\
    & = \frac{\Gamma(\alpha+\beta)}{\Gamma(\alpha)\Gamma(\beta)} (\sqrt{uv})^{\alpha-1}(1-\sqrt{uv})^{\beta-1} \frac{\Gamma(\alpha+\beta+\gamma)}{\Gamma(\alpha+\beta)\Gamma(\gamma)}\left(\sqrt{\frac{u}{v}}\right)^{\alpha+\beta-1}\left(1-\sqrt{\frac{u}{v}}\right)^{\gamma-1} \frac{1}{2v} \\
    & = \frac{\Gamma(\alpha+\beta+\gamma)}{\Gamma(\alpha)\Gamma(\beta)\Gamma(\gamma)} u^{\alpha-1} \left(\sqrt{\frac{u}{v}} -u\right)^{\beta-1} \left(1-\sqrt{\frac{u}{v}}\right)^{\gamma-1} \sqrt{\frac{u}{v}} \frac{1}{2v}\\
    \\
    & \text{Let $y = \frac{\sqrt{u/v}-u}{1-u}, 0 < y < 1, \quad v = \frac{u}{(y(1-u)+u)^2}$},\\
    & dv = \frac{-2u(1-u)}{(y(1-u)+u)^3} dy \\
    f_U(u) & = \frac{\Gamma(\alpha+\beta+\gamma)}{\Gamma(\alpha)\Gamma(\beta+\gamma)} u^{\alpha-1} (1-u)^{\beta+\gamma-2} \int_1^0 \frac{\Gamma(\beta+\gamma)}{\Gamma(\beta)\Gamma(\beta)} y^{\beta-1} (1-y)^{\gamma-1} \frac{(y(1-u)+u)^3}{2u} \frac{-2u(1-u)}{(y(1-u)+u)^3} dy \\
    & = \frac{\Gamma(\alpha+\beta+\gamma)}{\Gamma(\alpha)\Gamma(\beta+\gamma)} u^{\alpha-1} (1-u)^{\beta+\gamma-1} \int_0^1 \frac{\Gamma(\beta+\gamma)}{\Gamma(\beta)\Gamma(\beta)} y^{\beta-1} (1-y)^{\gamma-1} dy \\
    & = \frac{\Gamma(\alpha+\beta+\gamma)}{\Gamma(\alpha)\Gamma(\beta+\gamma)} u^{\alpha-1} (1-u)^{\beta+\gamma-1}\\
    \end{align*}
    So, $f_U(u)$ is the pdf of Gamma$(\alpha, \beta+\gamma)$, again.
    \end{enumerate}

    \section*{4.26}
    \begin{enumerate}[(a)]
    \item
    Since $X$ and $Y$ are independent random variables, we have,
    \[
    f(x,y) = f(x) f(y) = \frac{1}{\lambda} e^{-x/\lambda} \frac{1}{\mu} e^{-y/\mu}, \quad 0 < x, y < \infty
    \]
    
    If $X<Y, \quad Z=\min\{X,Y\} = X, 0 < z=x < y < \infty, \quad W = 1$. Therefore we will have,
    \begin{align*}
    f(z, w) &= P(Z=X \le z, W=1) = \int_0^z \int_x^\infty f(x,y)dydx \\
    & = \int_0^z \frac{1}{\lambda} e^{-x/\lambda} e^{-x/\mu} dx \\
    & = \frac{\mu}{\lambda + \mu} (1- e^{-\frac{\mu+\lambda}{\mu\lambda}z}). 
    \end{align*}
    Otherwise, if $Y<X$, we have,
    \begin{align*}
    F(z, w) &= P(Z=Y \le z, W=0) = \int_0^z \int_y^\infty f(x,y)dxdy \\
    & = \int_0^z \frac{1}{\mu} e^{-y/\mu} e^{-x/\lambda} dy \\
    & = \frac{\lambda}{\lambda + \mu} (1- e^{-\frac{\mu+\lambda}{\mu\lambda}z}). 
    \end{align*}
    \item
    \begin{align*}
    P(Z \le z) & = \sum_{i\in\{0, 1\}} P(Z \le z, W=i) \\
    & = 1- e^{-\frac{\mu+\lambda}{\mu\lambda}z}\\
    \\
    P(W = 1) & = P(X \le Y) = \intzi \int_0^y \frac{1}{\mu} e^{-y/\mu} \frac{1}{\lambda} e^{-x/\lambda} dx dy \\
    & = \frac{\mu}{\mu+\lambda}\\
    \\
    P(W = 0) & = P(Y \le X) = \intzi \int_0^x \frac{1}{\mu} e^{-y/\mu} \frac{1}{\lambda} e^{-x/\lambda} dy dx \\
    & = \frac{\lambda}{\mu+\lambda}\\
    \end{align*}
    We can identify that $P(Z\le z, W=i) = P(Z \le z)P(W = i), \quad i = 0, 1$. So $Z$ and $W$ are independent.
    \end{enumerate}

    \section*{4.28}
    \begin{enumerate}[(a)]
    \item
    Let $U = X/(X+Y), -\infty < u < \infty, \quad V=X+Y, -\infty < v < \infty$, so $X = UV, \quad Y=V(1-U)$.
    The Jacobian is,
    \[
    J = \begin{vmatrix}
    \frac{\partial uv}{\partial u} & \frac{\partial uv}{\partial v} \\
    \frac{\partial v(1-u)}{\partial u} & \frac{\partial v(1-u)}{\partial v} \\
    \end{vmatrix} = v
    \]

    Also, because $X$ and $Y$ are independent standard normal r.v.'s, 
    \begin{align*}
    f(x,y) & = \frac{1}{2\pi} e^{-\frac{x^2+y^2}{2}} \\
    f(u,v) & = f_{X,Y}(X=uv, Y=v(1-u))|J| \\
    & = \frac{1}{2\pi}e^{-\frac{v^2(2u^2-2u+1)}{2}}|v| \\
    f_{X/(X+Y)} = f_U(u) & = \frac{2}{2\pi} \intzi ve^{-\frac{v^2(2u^2-2u+1)}{2}} dv = \frac{1}{2\pi} \frac{1}{u^2-u+1/2} \\
    & = \frac{1}{\pi/2} \frac{1}{1+\left(\frac{(u-1/2)}{1/2}\right)^2}
    \end{align*}
    So, $X/(X+Y)$ has Cauchy$(1/2, 1/2)$ distribution.
    \item Let $Z = X/|Y|, -\infty < z \infty$. Then the cdf for $Z$ is,
    \begin{align*}
    F_Z(z) & = P(Z = X/|Y| \le z) = P(X \le z|Y|) \\
    & = \intii \int_{-\infty}^{z|y|} f_{X} (x) dx f_Y(y) dy \\
    & = \intii F_X(z|y|) f_Y(y) dy = 2 \intzi F_X(zy) f_Y(y) dy \\
    \\
    f_Z(z) & = \frac{d}{dz} 2 \intzi F_X(z) f_Y(y) dy \\
    & = 2 \intzi f_X(zy) y f_Y(y) dy 
    = \intzi \frac{1}{\pi} e^{-\frac{z^2y^2+y^2}{2}} dy \\
    & = \intzi \frac{1}{\pi} e^{-\frac{y^2(z^2+1)}{2}} dy \\
    & = \frac{1}{\pi}\frac{1}{1+z^2}
    \end{align*}
    So, $X/|Y|$ has a Cauchy$(1,1)$ distribution.
    \end{enumerate}

    \section*{4.30}
    \begin{enumerate}
    \item[(b)] We have shown that,
    \[
    f(x, y) = \frac{1}{\sqrt{2\pi} x} e^{-\frac{(y-x)^2}{2x^2}}.
    \]
    Now let $U=Y/X, -\infty < u < \infty; \quad V=X, 0 < v < 1$, then $Y = UV, X = V$. The Jacobian is,
    \[
    J = \begin{vmatrix}
    \frac{\partial uv}{\partial u} & \frac{\partial uv}{\partial v} \\
    \frac{\partial v}{\partial u} & \frac{\partial v}{\partial v} \\
    \end{vmatrix} = -v
    \]

    Therefore,
    \begin{align*}
    f(u,v) & = f(X=v, Y=uv)|J| \\
    & = \frac{1}{\sqrt{2\pi}v} v e^{-\frac{(uv-v)^2}{2v^2}} \\
    & = \frac{1}{\sqrt{2\pi}} e^{-\frac{(u-1)^2}{2}}
    \end{align*}
    And we can easily see that this joint density can be factorized into $g_1(u) = \frac{1}{\sqrt{2\pi}} e^{-\frac{(u-1)^2}{2}}, \quad g_2(v) = 1$. So, $Y/X$ and $X$ are independent.
    \end{enumerate}

    \section*{4.42}
    Given that $X$ and $Y$ are independent random variables, we could have,
    \begin{align*}
    \rho & = \frac{Cov(XY, Y)}{\sqrt{Var XY Var Y}} = \frac{EXY^2 - EXYEY}{\sqrt{(E(XY)^2 - (EXY)^2)Var Y}} \\
    & = \frac{EXEY^2 - EX(EY)^2}{\sqrt{(EX^2EY^2 - (EX)^2(EY)^2)Var Y}} \\
    & = \frac{\mu_X (\sigma_Y^2 + \mu_Y^2) - \mu_X \mu_Y^2}{\sqrt{\big((\sigma^2_X + \mu_X^2)(\sigma^2_Y + \mu_Y^2)-\mu_X^2\mu_Y^2\big)\sigma^2_Y}} \\
    & = \frac{\mu_X\sigma_Y}{\sqrt{\sigma^2_X\sigma^2_Y + \mu_X^2\sigma^2_Y + \mu_Y^2\sigma^2_X}}
    \end{align*}
    \section*{4.45}
    \begin{enumerate}[(a)]
    \item The pdf of a bivariate normal distribution given parameter $\mu_X, \mu_Y, \sigma_X^2, \sigma^2_Y, \rho$ is,
    \[
    f(x, y) = \frac{1}{2\pi\sigma_X\sigma_Y \sqrt{1-\rho^2}} \exp\left(-\frac{1}{2(1-\rho^2)} \left(\frac{(x-\mu_X)^2}{\sigma_X^2} - \frac{2\rho(x-\mu_X)(y-\mu_Y)}{\sigma_X \sigma_Y} + \frac{(y-\mu_Y)^2}{\sigma_Y^2}\right) \right)
    \]
    So, the marginal pdf $f_X(x)$ is,
    \begin{align*}
    f(x) & = \intii f(x,y) dy \\
    & = \frac{1}{\sqrt{2\pi}\sigma_X} \exp\left(\frac{1}{2(1-\rho^2)} (1-\rho^2)\frac{(x-\mu_X)^2}{\sigma_X^2}\right) \intii \frac{1}{\sqrt{2\pi}\sigma_Y \sqrt{1-\rho^2}} \exp \left(\frac{1}{1-\rho^2} \left( \frac{y-\mu_Y}{\sigma_Y} - \rho \frac{x-\mu_X}{\sigma_X} \right)^2 \right) dy \\
    & = \frac{1}{\sqrt{2\pi}\sigma_X} e^{-\frac{(x-\mu_X)^2}{2\sigma_X^2}} \intii \frac{1}{\sqrt{2\pi}\sigma_Y \sqrt{1-\rho^2}} e^{-\cfrac{\left(y-(\mu_Y +\rho \frac{\sigma_Y(x-\mu_X)}{\sigma_X}) \right)^2}{2(1-\rho^2)\sigma_Y^2}  } dy \\
    \end{align*}
    Notice that the integrand in the second term is a normal density, so it will integrate to 1. Thus, 
    \[
    f(x) = \frac{1}{\sqrt{2\pi}\sigma_X} e^{-\frac{(x-\mu_X)^2}{2\sigma_X^2}}
    \]
    \emph{i.e.} $X$ has marginal distribution of $N(\mu_X, \sigma_X^2)$.

    Since $X$ and $Y$ are equivalent and interchangable in $f(x,y)$, so the marginal distribution of $Y$ should be $N(\mu_Y, \sigma^2_Y)$.
    \item 
    \begin{align*}
    f(Y|X=x) & = \frac{f_{X,Y}(x,y)}{f_X(x)} \\
    & = \cfrac{\frac{1}{2\pi\sigma_X\sigma_Y \sqrt{1-\rho^2}} \exp\left(-\frac{1}{2(1-\rho^2)} \left(\frac{(x-\mu_X)^2}{\sigma_X^2} - \frac{2\rho(x-\mu_X)(y-\mu_Y)}{\sigma_X \sigma_Y} + \frac{(y-\mu_Y)^2}{\sigma_Y^2}\right) \right)}{\frac{1}{\sqrt{2\pi}\sigma_X}\exp\left(-\frac{(x-\mu_X)^2}{2\sigma_X^2}\right)} \\
    & = \frac{1}{\sqrt{2\pi}\sigma_Y\sqrt{1-\rho^2}} e^{-\cfrac{1}{2(1-\rho^2)} \left(\rho \cfrac{(x-\mu_X)^2}{\sigma_X^2} - \cfrac{2\rho(x - \mu_X)(y - \mu_Y)}{\sigma_X \sigma_Y} + \cfrac{(y-\mu_Y)^2}{\sigma_Y^2}\right)} \\
    & = \frac{1}{\sqrt{2\pi}\sigma_Y\sqrt{1-\rho^2}} e^{-\cfrac{1}{2(1-\rho^2)} \left(\rho \cfrac{\sigma_Y^2(x-\mu_X)^2}{\sigma_Y^2\sigma_X^2} - \cfrac{2\rho\sigma_Y(x - \mu_X)(y - \mu_Y)}{\sigma_X \sigma_Y^2} + \cfrac{(y-\mu_Y)^2}{\sigma_Y^2}\right)} \\
    & = \frac{1}{\sqrt{2\pi}\sigma_Y\sqrt{1-\rho^2}} e^{-\cfrac{\left((y-\mu_Y) - \frac{\rho\sigma_Y (x-\mu_X)}{\sigma_X}\right)^2}{2\sigma^2(1-\rho^2)}}
    \end{align*}
    And this is the pdf for $N(\mu_Y + \rho(x-\mu_X)\frac{\sigma_Y}{\sigma_X}, \sigma^2_Y(1-\rho^2))$.
    \end{enumerate}
\end{document}
